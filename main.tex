\documentclass{article}
\usepackage{graphicx} % Required for inserting images
\usepackage{amsmath}
\usepackage{amssymb} %THP : nécessaire pour avoir \mathbb
\usepackage{amsthm} %THP : pour avoir des environnements adaptés pour les définitions etc. 

\theoremstyle{definition}
\newtheorem{lemme}{Lemme}
\newtheorem{definition}{Définition}
\newtheorem{theoreme}{Théorème}
\newtheorem{preuve}{Preuve}
\newtheorem{notation}{Notation}
\newtheorem{remarque}{Remarque}

\title{Latex}
\author{Amaury Briand}
\date{July 2024}

\begin{document}

\maketitle

%THP : J'ai ajouté des environnements pour les définitions, lemmes, etc.
%THP : Note que \text ne s'utilise que très rarement et toujours dans des équations ($$ ou environnement equation, equation*, align ou align*).

\section{Marche aléatoire isotrope dans $\mathbb{Z}^d$}

\begin{notation}
On note $X_{n}$ une suite de variables aléatoires tel que pour tout n$\in {N}^{*}$ P($X_{n}$ = 1) = 1/2 et P($X_{n}$ = -1) = 1/2. On note $S_{n} = \sum_{k=1}^{2n} X_{k}$ ainsi $S_{n}$ représente la "position" de la marche aléatoire après n pas. 
La marche aléatoire est donc à son origine à la n-ième étape lorsque $S_{n}$ = 0. 
\end{notation}

\begin{definition}
    On dit que la marche aléatoire est récurrente si
    elle revient pour sur à 0.
\end{definition}

\begin{lemme}
    En notant Y = $|$n$\in {N}^{*}$, $S_{n}$ = 0$|$, la marche est récurrente si et seulement si E(Y) = +$\infty$.
\end{lemme}

\begin{preuve}
     Supposons la marche récurrente on dispose alors de n dans $N^{*}$ tel que  $S_{n}$ = 0 après la n-ième étape la marche est dans la même situation que à l'étape 0. Elle va donc nécessairement revenir à 0 une nouvelle fois après n. La marche revient donc une infinité de fois à 0. D'où E(Y) = +$\infty$
     Réciproquement supposons E(Y) = $\infty$ montrons que la marche revient pour sur en 0.
     Notons p la probabilité que le marcheur revienne à 0 et supposons par l'absurde que p < 0.
     Alors la probabilité que le marcheur ne revienne 
     pas à l'origine est 1-p. D'où P(V = 0) = 1-p. 
     Plus généralement pour k dans N, P(V = k) = p^{k}, montrons ce résultat par récurrence.
     Pour V=0, on a montré qu'il est valide. Supposons le valide pour m dans N. Alors P(V = m+1) = P(V = m) * p donc P(V = m+1) = p^{m+1} (1-p). D'où le résultat par récurrence.
     Par le théorème de transfert on a
\begin{align*}
E(V) &= \sum_{k}^{+\infty} k P(X = k)
&= \sum_{k}^{+\infty} k (1 - p) p^k
&= p (1 - p) \sum_{k = 1}^{+\infty} k p^{k - 1}
En posant \(f^{k} (x) \mapsto x^{k-1} \), on a
= p (1 - p) \sum_{k = 1}^{+\infty} f'_k (p)
=p (1 - p) (\sum_{k = 1}^{+\infty} f_k (p))'
= p (1 - p) ( \frac{p}{(1 - p)} )'
= p (1 - p)  \frac{1-p+p}{(1 - p)^2} 
= \frac{p}{1 - p} < {+\infty}
\end{align*}
Or E(V) = {+\infty} on a donc une contradiction, d'où p=1 et donc la marche est récurrente.
\end{preuve}


\section{Cas en 1 dimension}
\begin{remarque}
On s'intéresse uniquement à des marches ayant un nombre pair de pas, le retour à l'origine au bout d'un nombre impair de pas étant impossible.
\end{remarque}

\begin{align*}
P(S_{2n} = 0) &= |P( \sum_{k=1}^{2n} X_k = 0 )| \\
 &= \binom{2n}{n} (\frac{1}{2})^n (\frac{1}{2})^n \\
 &= \frac{2n!}{(n!)^2} (\frac{1}{2})^2n \\
 &= \frac{\sqrt{4 \pi n} (\frac{2n}{e})^{2n}}{(\sqrt{2 \pi n} (\frac{n}{e})^n)^{2}} (\frac{1}{2})^{2n}\\
 &= \frac{1}{\sqrt{\pi n}}\\
\end{align*}
Or $\sum_{n=1}^{+\infty} \frac{1}{\sqrt{\pi n}} = +\infty$ \\
Donc $\sum_{n=1}^{+\infty} P(S_{2n} = 0) = +\infty$
Or d'après le théorème de transfert E(Y) = $\sum_{n=1}^{+\infty} P(S_{2n} = 0)$ \\
Par le lemme établi précedemment, dans Z la marche aléatoire isotrope est récurrente.

\section{Vérification numérique}

\section{Cas en dimension 2}


$X_{k}$  représente la direction prise à la k-ième étape ainsi


$X_k$ = 
\begin{array}
B si bas \\
G si gauche \\
D si droite
\end{array}
\\
On note card(A) = $|k \in {N}^{*},X_{k} = A|$ pour A = G,D,H ou B
\\
\begin{align*}
P(S_{2n} = 0) &= P ( card(H) = card(B) \land  card(G) =|card(D) )\\  
&= P( \bigcup_{k=0}^{n} \bigcup_{\substack{P \subset [0, 2N] \cap \mathbb{N},  |P| = k}} card(H) = card(B) = k \land card(G) = card(D) = n-k) \\
&= \sum_{k=0}^{n} \binom{2n}{k} \frac{1}{4^{k}} \binom{2n-k}{k} \frac{1}{4^{k}} \binom{2n-2k}{n-k} \frac{1}{4^{n-k}} \frac{1}{4^{n-k}} \\
&= \frac{1}{4^{2n}} \sum_{k=0}^{n}  \frac{(2n)!}{k!(2n-k)!}\cdot \frac{(2n-k)!}{k!(2n-2k)!} \cdot \frac{(2n-2k)!}{((n-k)!)^2} \\
 &= \frac{1}{4^{2n}} \sum_{h=0}^{2n} \frac{(2n)!}{(h!)^2 (n-h)!^2}\\
 &= \frac{1}{4^{2n}} \sum_{h=0}^{2n} \frac{(2n)!}{(n!)^2} \binom{n}{h}^2\\
 &= \frac{1}{4^{2n}} \frac{(2n)!}{(n!)^2} \sum_{h=0}^{2n}  \binom{n}{h}^2
\end{align*}

\textbf{Calcul de} $\sum_{k=0}^{n} \binom{n}{k}^2$ : \\

\begin{equation*}
(1+x)^{2n} = \sum_{k=0}^{2n} \binom{2n}{k} x^k
\end{equation*}
 or 
 \begin{equation*}
 (1+x)^{2n} = ( \sum_{k=0}^{n} \binom{n}{k} x^k )^2
\end{equation*}
en identifiant le terme en $x^{2n}$ on obtient la relation :

\begin{align*}
\binom{2n}{n} x^n &=  \sum_{k=0}^{n} \binom{n}{k} \binom{n}{n-k} x^k x^{n-k} \\
 &= \sum_{k=0}^{n} \binom{n}{k} \binom{n}{n-k} x^n \\
  \end{align*} 
 Ainsi \\
$ \sum_{k=0}^{n} \binom{n}{k}^2 &= \binom{2n}{n}$ \\
 D'ou \\
$ &= \frac{1}{\pi n} \\
 \sum_{n=1}^{+\infty} \frac{1}{\pi n} &= + \infty \\$

De même que précedemment avec le théorème de transfert on obtient \\
E(Y) = $\sum_{n=1}^{+\infty} P(S_{2n} = 0)$ \\
Donc la marche est récurrente en dimension 2.

\section{Temps moyen de premier retour en 1D}

\section{Vérification numérique}

\section{Transience en dimension supérieure ou égale à 3}

On s'intéresse au calcul de l'espérance de Y = $|$n$\in {N}^{*}$, $S_{n}$ = 0$|$ en dimension d.
On peut obtenir que la probabilité que la marche soit à la position 0 après 2n étape est un grand O$(n^{d/2})$ ainsi on a 
Y = $\sum_{n=0}^{+inf}$ $P(S_{n})$ qui converge pour d ≥ 3, ainsi par le lemme 1, la marche est transiente.

Note : En dimension 1 et 2, cette propriété reste valable puisque la série $\sum_{n=0}^{+inf}$ $P(S_{n})$ est alors divergente et donc Y est infini et la marche est récurrente.

\end{document}
